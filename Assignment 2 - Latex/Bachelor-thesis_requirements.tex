\documentclass[a4paper,12pt]{article}
\usepackage{biblatex}
\addbibresource{referencer.bib}
\title{Latex Bachelor Thesis Requirements}
\date{2021\\March}
\author{Frederikke Simone Koefoed Nilsson, Emil Skovbo, Frederik Lassen}


\begin{document}
\maketitle
\thispagestyle{empty}
\clearpage
\pagenumbering{arabic}

\tableofcontents
\clearpage

\section{Introduction}
Our findings in searching for requirements for a bachlor thesis.
\section{Krav til bachelor projekt}

Det maksimale antal sider for bachelor afhandlingen er givet af en formular: \\ \\
$maxPageCount = 40 + 20 * numberOfStudents $ \\ \\
Men officielt er den ingen minimum størrelse. \\ \\
Afhandlingen skal indeholde en grundig beskrivelse af arbejdet som den studerende har udført, samt en evaluering og reflektion af dette. \\ \\
Afhandlingen skal være enten på engelsk eller dansk.\\ \\
Forslået sektioner i afhandlingen er som følgende:

\begin{itemize}

\item Projekt titel, forside
\item Abstrakt
\item Introduktion
\item Teknologier
\item Krav for specifikationer og design
\item Udvikling og implementering
\item Konklusioner
\item Bilag
\item Fremvisning

\end{itemize}

\clearpage

\section{Læringsmål}

Det afsluttende bachelorprojekt skal dokumentere, at uddannelsens afgangsniveau er opnået, jf. bilag 1 i BEK for professionsbacheloruddannelsen i softwareudvikling: \\
\\
\textbf{Viden} \\ \\
Den uddannede har viden om:
\begin{itemize}
\item den strategiske rolle af test i systemudvikling
\item globalisering af softwareproduktion
\item systemarkitektur og forståelse af dens strategiske betydning for virksomhedens forretning
\item anvendt teori og metode samt udbredte teknologier inden for domænet
\item sammenhænge mellem anvendt teori, metode og teknologi og kan reflektere over disses egnethed i forskellige situationer
\end{itemize}
\vspace{1cm}
\textbf{Færdigheder}  \\ \\
Den uddannede kan:
\begin{itemize}
\item håndtere planlægning og gennemførelse af test af større IT-systemer
\item indgå professionelt i samarbejde omkring udvikling af store systemer ved anvendelse af udbredte metoder og teknologier
\item sætte sig ind i nye teknologier og standarder til håndtering af integration mellem systemer

\item gennem praksis udvikle egen kompetenceprofil fra primært at være en backend-udviklerprofil til at varetage opgaver som
systemarkitekt
\item håndtere fastlæggelse og realisering af en såvel forretningsmæssig som teknologisk hensigtsmæssig arkitektur for store systemer

\end{itemize}
\vspace{1cm}
\textbf{Kompetencer} \\
\\
Den uddannede kan:
\begin{itemize}
\item håndtere planlægning og gennemførelse af test af større IT-systemer
\item indgå professionelt i samarbejde omkring udvikling af store systemer ved
anvendelse af udbredte metoder og teknologier
\item sætte sig ind i nye teknologier og standarder til håndtering af integration
mellem systemer
\item gennem praksis udvikle egen kompetenceprofil fra primært at være en
backend-udviklerprofil til at varetage opgaver som systemarkitekt
\item håndtere fastlæggelse og realisering af en såvel forretningsmæssig som
teknologisk hensigtsmæssig arkitektur for store systemer
\end{itemize}

\section{Sources}
$https://www.cphbusiness.dk/media/1177/pba_soft_cba_studieordning.pdf$
\\ \\
$https://datsoftlyngby.github.io/soft2020fall/resources/bbe51cf2-bachelorProject.pdf$



\printbibliography


\end{document}
